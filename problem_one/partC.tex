\documentclass[../report/main.tex]{subfiles}
 
\begin{document}

% The asterix after \subsection disables section numbering
\subsection*{Problem 1 Part C}
Determine the number of refrigerators to be shipped from plants to warehouses, and then warehouses to retailers to minimize the cost.  \textcolor{blue}{For part C warehouse 2 is limited to just 100 units per week.  Changes to the problem statement or solution relative to Problem 1 Part A are highlighted in blue.}

\subsection*{Solution}
\begin{itemize}
	\item In all 1000 units will travel through the network at a minimum cost of \textcolor{blue}{\$18300}.
	\newline
	\item Ship 150 units from plant \#1 to warehouse \#1 at a cost of \$1500.
	\item Ship \textcolor{blue}{350} units from plant \#2 to warehouse \#1 at a cost of \textcolor{blue}{\$3850}.
	\item Ship \textcolor{blue}{100} units from plant \#2 to warehouse \#2 at a cost of \textcolor{blue}{\$800}.
	\item Ship   \textcolor{blue}{0} units from plant \#3 to warehouse \#2 at a cost of \textcolor{blue}{\$   0}.
	\item Ship \textcolor{blue}{250} units from plant \#3 to warehouse \#3 at a cost of \textcolor{blue}{\$2250}.
	\item Ship 150 units from plant \#4 to warehouse \#3 at a cost of \$1200.
	\newline
	\item Ship 100 units from warehouse \#1 to retailer \#1 at a cost of \$ 500.
	\item Ship 150 units from warehouse \#1 to retailer \#2 at a cost of \$ 900.
	\item Ship 100 units from warehouse \#1 to retailer \#3 at a cost of \$ 700.
	\item \textcolor{blue}{Ship 150 units from warehouse \#1 to retailer \#4 at a cost of \$1500}.
	\item Ship  \textcolor{blue}{50} units from warehouse \#2 to retailer \#4 at a cost of \textcolor{blue}{\$ 400}.
	\item Ship  \textcolor{blue}{50} units from warehouse \#2 to retailer \#5 at a cost of \textcolor{blue}{\$ 500}.
	\item \textcolor{blue}{Ship 150 units from warehouse \#3 to retailer \#5 at a cost of \$1800.}
	\item Ship 150 units from warehouse \#3 to retailer \#6 at a cost of \$1800.
	\item Ship 100 units from warehouse \#3 to retailer \#7 at a cost of \$ 600.
	\newline
	\item 150 total units will leave plant \#1 (capacity is 150).
	\item 450 total units will leave plant \#2 (capacity is 450).
	\item 250 total units will leave plant \#3 (capacity is 250).
	\item 150 total units will leave plant \#4 (capacity is 150).
	\newline
	\item \textcolor{blue}{500} total units will enter warehouse \#1, \textcolor{blue}{500} units will leave.
	\item \textcolor{blue}{100} total units will enter warehouse \#2, \textcolor{blue}{100} units will leave.
	\item \textcolor{blue}{400} total units will enter warehouse \#3, \textcolor{blue}{400} units will leave.
	\newline
	\item 100 total units will enter retailer \#1 (demand is 100).
	\item 150 total units will enter retailer \#2 (demand is 150).
	\item 100 total units will enter retailer \#3 (demand is 100).
	\item 200 total units will enter retailer \#4 (demand is 200).
	\item 200 total units will enter retailer \#5 (demand is 200).
	\item 150 total units will enter retailer \#6 (demand is 150).
	\item 100 total units will enter retailer \#7 (demand is 100).
\end{itemize}
\subsection*{Linear Program Formulation}
\begin{enumerate}[1.]
	\item Overall idea of problem
	\begin{itemize}
		\item Refrigerators moving from $n=4$ plants to $q=3$ warehouses to $m=7$ retailers.
		\item Not all plants deliver to all warehouses.
		\item Not all warehouses deliver to all retailers.
		\item Costs of shipping from plants to warehouses vary by pair.
		\item Costs of shipping from warehouses to retailers vary by pair.
		\item Each plant has a capacity in terms of number of refrigerators it can supply.
		\item Each retailer has a capacity in terms of number of refrigerators it demands.
		\item\textcolor{blue}{Warehouse 2 is limited to just 100 units in and out per week.}		
	\end{itemize}
	\item What is the goal?  What are you trying to achieve?
	\begin{itemize}
		\item Unchanged from part A.
	\end{itemize}
	\item Identify variables
	\begin{itemize}
		\item Unchanged from part A.
	\end{itemize}
	\item Identify constraints
	\begin{itemize}
		\item All constraints from part A remain in effect with the addition of two new constraints:
		\item\textcolor{blue}{$np_{12} + np_{22} + np_{32} + np_{42} = 100$}		
		\item\textcolor{blue}{$nw_{23} + nw_{24} + nw_{25} + nw_{26} = 100$}		
	\end{itemize}
	\item Identify inputs and outputs that you can control
	\begin{itemize}
		\item Unchanged from part A.
	\end{itemize}
	\item Specify all quantities mathematically
	\begin{itemize}
		\item Unchanged from part A.
	\end{itemize}
	\item Check the model for completeness and correctness
	\begin{itemize}
	\item All variables are positive.
	\end{itemize}
\end{enumerate}
\subsection*{ Matlab Code}
\textcolor{blue}{Code minimally changed from part A.  Only changes are 2 additional constraints (16 total equations) in the linear equality matrix and vector.  Identical code from part A is not shown below (to save space).}
\lstinputlisting{../problem_one/partC_changes_fromA.m}
\end{document}